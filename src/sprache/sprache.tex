
\section{Sprache}

\textbf{Definition:} 
Verschiedene "arten" von Sprache, abhängig von verschieden Aspekten.

\textbf{Beispiele:}
\begin{itemize}
    \item Regiolekt: regionale Unterschiede
    \item Idiolekt: indiviuelle Sprachverwendung
    \item Genderlekt: Männer und Frauen reden anders
    \item Fachsprache: spezialisierte Sprache eines Fachgebiets; präzise Kommunikation unter Experten
    \item Dialekte: Bayrisch; Schwäbisch
    \item Soziolekte: Jugendsprache; Bildungssprache
\end{itemize}

\textbf{Funktion:}
\begin{itemize}
    \item Identitätsstiftung
    \item Gruppenzugehörigkeit - soziale Abgrenzung
\end{itemize}


\subsection{Sprachwandel}

\textbf{"Gesetz wie sich Sprache verändert:"}
\begin{itemize}
    \item was am besten verstanden wird
    \item was als sprachliche Ökonomie wahrgenommen wird
    \item womit man sich am besten durchsetzen oder imponieren kann
\end{itemize}


\subsubsection{Test}

