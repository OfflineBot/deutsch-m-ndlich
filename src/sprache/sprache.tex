
\section{Sprache}


\subsection{Sprachvariationen}

\point{Definition:} 
Verschiedene "arten" von Sprache, abhängig von verschieden Aspekten.

\point{Beispiele:}
\begin{itemize}
    \item Regiolekt: regionale Unterschiede
    \item Idiolekt: indiviuelle Sprachverwendung
    \item Genderlekt: Männer und Frauen reden anders
    \item Fachsprache: spezialisierte Sprache eines Fachgebiets; präzise Kommunikation unter Experten
    \item Dialekte: Bayrisch; Schwäbisch
    \item Soziolekte: Jugendsprache; Bildungssprache
\end{itemize}

\point{Funktion:}
\begin{itemize}
    \item Identitätsstiftung
    \item Gruppenzugehörigkeit - soziale Abgrenzung
\end{itemize}


\subsubsection{Sprachwandel}

\point{"Gesetz wie sich Sprache verändert:"}
\begin{itemize}
    \item was am besten verstanden wird
    \item was als sprachliche Ökonomie wahrgenommen wird
    \item womit man sich am besten durchsetzen oder imponieren kann
\end{itemize}

\point{These 1:}
Sprache als natürlicher Organismus \\
$\rightarrow$ Wandel ohne bewusste Einflussnahme

\point{These 2:}
Sprache verändert sich nur durch Gebrauch

\point{These 1 + These 2:}
$\Rightarrow$ Sprachwandel (Synthese)


\subsection{Politische Kommunikation}

\point{Ziel:}
Meinung beeinflussen um Zustimmung (Stimmen) zu gewinnen \\
$\rightarrow$ Macht

\point{Merkmale:}
\begin{itemize}
    \item Rhetorische Mittel: Methaphern, Wiederholungen $\rightarrow$ Polarisieren
    \item Framing: Einordung von Themen in einen bestimmten Rahmen ("Klimakrise" vs "Klimahysterie")
    \item Populismus: Vereinfachung, Emotionalisierung, "Wir gegen die"
    \item Sprachlenkung: Begriffe bewusst wählen oder vermeiden (BILD Zeitung)
\end{itemize}


\subsection{Sprache-Denken-Wirklichkeit}

\subsubsection{Sapir-Whorf-Hypothese}

\point{These:}
Die Sprache beeinflusst, wie wir denken und die Welt wahrnehmen. \\
$\rightarrow$ Sprache bestimmt oder beeinflusst denken

\point{Beispiele:}
Inuits haben viele Wörter für Schnee $\rightarrow$ differenzierte Wahrnehmung für Schnee

\point{Kritik:}
Wurde bereits Widerlegt \\
$\rightarrow$ Denken ist auch ohne Sprache möglich

\point{Relevanz:}
Sprache schafft Realitäten, z.B. durch Begriffsprägung in Politik und Medien (z.B. "Heizungshammer" von der BILD)


